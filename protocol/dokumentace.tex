\documentclass[a4paper,11pt]{article}

\usepackage[left=2cm,text={17cm, 24cm}, top=3cm]{geometry}
\usepackage[czech]{babel}
\usepackage[utf8]{inputenc}
\usepackage{times}
\begin{document}
\begin{titlepage}
\begin{center}
\Huge
\textsc{Vysoké učení technické v~Brně\\ \huge Fakulta informačních technologií}\\
\vspace{\stretch{0.382}}
\Huge Fyzikální optika \\ 
\LARGE Hubbleův a Webbův teleskop - popis optických systémů  \\
\vspace{\stretch{0.109}}
semestrální projekt \\
\vspace{\stretch{0.309}}
\today
\vspace{\stretch{0.209}}
\end{center}
\Large Zdeněk Biberle - xbiber00@stud.fit.vutbr.cz \\ Josef Řídký - xridky00@stud.fit.vutbr.cz
\end{titlepage}
\newpage

\tableofcontents

\newpage

\Large\noindent{\bf Abstrakt}\\
\normalsize\noindent\\ Tento projekt popisuje historii, vlastnosti a~popis optických soustav již používaného Hubbleova teleskopu a~plánovaného teleskopu Jamese Webba. V~první části je čtenář seznámen s~Hubbelovým teleskopem a~v~druhé části s~teleskopem Jamese Webba. Součástí tohoto projektu je i~animace popisující optické soustavy obou vesmírných teleskopů.\\
\vspace{\stretch{0.8}}\\
\Large\noindent {\bf Klíčová slova}\\
\normalsize\noindent teleskop, optické systémy, čočky, vesmír, Hubble, James Webb, NASA\\
\vspace{\stretch{0.2}}
\newpage


\section{Úvod}
Člověk už od pradávna vzhlížel ke hvězdám a~vesmírným tělesům ve snaze je zdokumentovat, zaznamenat jejich pohyb či zajistit si jejich příznivé naklonění. V~posledních letech se však lidstvu díky technologickému pokroku podařilo udělat obrovský skok vpřed v~pozorování a~dokumentování naší galaxie a~vesmíru kolem nás. Díky rozmachu vesmírných programů se podařilo umístit přímo do kosmu speciální dužice a~teleskopy díky nimž jsme opět o~krok blíž k poznání tajů vesmíru. V~tomto semestrálním projektu se zabýváme historií, vlastnostmi a~popisem optických soustav dvou vesmírných teleskopů. Prvním teleskopem je Hubbleův teleskop, který je na oběžné dráze v~provozu již 25 let a~pomohl nám objasnit a~objevit mnoho záhad vesmíru. Druhým popsaným teleskopem je teleskop Jamese Webba, který ještě ve vesmíru není, avšak již existuje jeho model ve skutečné velikosti. Konstrukce finálního teleskopu je již v~plném proudu, aby mohl v~roce 2018 započít svoji misi. 

\section{Hubbleův vesmírný dalekohled}
Hubbleův vesmírný teleskop (známý také pod svou zkratkou HST, či pouze jako Hubble) je teleskop, který byl vypuštěn na oběžnou dráhu v roce 1990 \cite{nasaHubbleChronology}. Od té doby byly k teleskopu vyslány čtyři servisní mise \cite{hubbleSiteServicingMissions}, které postupně prodlužovaly až do původně očekávaného ukončení činnosti v roce 2015, NASA nyní ovšem tvrdí, že HST by mohl vydržet v provozu do roku 2018 \cite{hubbleUntil2018}, tedy do vypuštění vesmírného teleskopu Jamese Webba.


\subsection{Historie}
\subsection{Popis konstrukce}
\subsection{Optický systém}


\section{Vesmírný teleskop Jamese Webba}

Druhým teleskopem, který v~tomto projektu představíme, je vesmírný teleskop Jamese Webba. Vesmírný teleskop Jamese Webba (anglicky James Webb Space Telescope, dále v textu jen JWST) je plánovaná vesmírná observatoř, která by měla být vypuštěna v~říjnu 2018. JWST by měl nahradit dosluhující Hubbleův teleskop. V jednotlivých podkapitolách budeme postupně rozebírat jeho zvláštnosti a~parametry. V první řadě se ale zaměříme na jeho historii a~pojmenování.

\subsection{Představení teleskopu}
JWST je plánovaná velká vesmírná observatoř optimalizovaná pro příjem infračerveného vlnového záření, pomocí kterého budou doplněny a~prohloubeny obejvy, které byly uskutečněny Hubbleovým teleskopem.
Díky větší vlnové délce bude JWST schopen pozorovat formování prvních galaxií při utváření vesmíru a dále bude schopen pozorovat formování galaxií, které jsou ukryré za mračny vesmírného prachu.

Původní název JSWT byl Vesmírný teleskop nové generace (anglicky Next Generation Space Telescope, zkráceně NGST). Přívlastek 'nové generace' získal proto, že JWST bude stavět a pokračovat ve vědním průzkumu, který započal Hubbleův vesmírný teleskop. Oběvy, které byly učiněny Hubbleovým teleskopem, ale i ostatními teleskopy, zapříčinily revoluci v astronomii a vyvstaly nové otázky, na jejichž zodpovězení bylo zapotřebí nových, odlišných a výkonějšších teleskopů. 
JWST je také teleskopem nové generace i~po stránce technologické a~to díky použití nových technologií jako například odlehčeného, rozvinovacího primárního zrcadla, které bude určovat směr pro budoucí mise.
Vesmírný teleskop nové generace byl 10. září 2002 pojmenován po Jamesi E. Webbovi.
 
\subsubsection{James E. Webb}
\subsection{Popis konstrukce}
\subsection{Podsekce 3}

\section{Závěr}

\newpage

\bibliographystyle{czechiso}
\def\refname{Použitá literatura}
\bibliography{dokumentace}

\end{document}
